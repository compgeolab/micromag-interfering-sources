The first step in scientific data acquisition often involves analyzing entire samples, providing only a general characterization of the material. Enhancing data acquisition by improving spatial resolution and isolating the underlying phenomena contributing to the overall signal has become a central direction in various scientific fields. In paleomagnetism, this advancement is now possible with the advent of magnetic microscopy (MM), whose high spatial resolution and magnetic moment sensitivity allow for imaging samples at the mineral grain scale. In this study, we aim to obtain reliable paleomagnetic directions using only MM data. To achieve this, we apply Euler deconvolution to solve the linear problem and mitigate the non-uniqueness associated with inversion. As an additional step, we refine the recovered parameters using a nonlinear inversion and remove interfering signals between sources to minimize noise. This algorithm was applied to both synthetic and real data and compared to its predecessor. The results from synthetic data demonstrate that this new approach is able to detect weaker sources and produce more accurate grain-level results, which in turn leads to larger datasets and improved statistical characterization of the sample. For real data, we observe that the iterative method was significantly more efficient than its predecessor, successfully retrieving the natural remanent magnetization direction of a basaltic sample with \ang{3} accuracy. This represents a significant step forward in applying MM data to paleomagnetic studies.