Scientists often study entire samples to understand their overall properties, but this approach can miss important details. To get a clearer picture, researchers are improving methods that focus on smaller regions of a sample. In paleomagnetism, a field that studies the Earth's ancient magnetic field, magnetic microscopy allows scientists to examine tiny areas with high precision. In this study, we use magnetic microscopy data to determine the direction of magnetization in samples. To do this, we apply a mathematical method called Euler deconvolution, which helps solve complex calculations and reduce uncertainty. We also refine our results with an additional step that improves accuracy and removes unwanted signals. We tested this approach on both simulated and real data. Our results show that this new method can detect weaker magnetic sources and accurately determine the direction of magnetization. When applied to real samples, it successfully identified their original magnetic direction. This represents an important step in using magnetic microscopy for paleomagnetic research.